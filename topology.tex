\documentclass{article}

\usepackage{ctex}
\usepackage{tikz}
\usetikzlibrary{cd}

\usepackage{amsthm}
\usepackage{amsmath}
\usepackage{amssymb}

\usepackage{unicode-math}


\usepackage[textwidth=18cm]{geometry} % 设置页宽=18

\usepackage{blindtext}
\usepackage{bm}
\parindent=0pt
\setlength{\parindent}{2em} 
\usepackage{indentfirst}


\usepackage{xcolor}
\usepackage{titlesec}
\titleformat{\section}[block]{\color{blue}\Large\bfseries\filcenter}{}{1em}{}
\titleformat{\subsection}[hang]{\bfseries}{}{0em}{}
%\setcounter{secnumdepth}{1} %section 序号

\newtheorem{theorem}{Theorem}[section]
\newtheorem{lemma}[theorem]{Lemma}
\newtheorem{corollary}[theorem]{Corollary}
\newtheorem{proposition}[theorem]{Proposition}
\newtheorem{example}[theorem]{Example}
\newtheorem{definition}[theorem]{Definition}
\newtheorem{remark}[theorem]{Remark}
\newtheorem{exercise}{Exercise}[section]

\newcommand*{\xfunc}[4]{{#2}\colon{#3}{#1}{#4}}
\newcommand*{\func}[3]{\xfunc{\to}{#1}{#2}{#3}}

\begin{document}
\title{Topology}
\author{枫聆}
\maketitle

\tableofcontents
\section{写在最前面}

拓扑对我来说是新名字,我对它几乎一无所知,虽然我嘴上总是吵吵着代数拓扑是我的终极目标\verb|(~ o ~)~zZ|,终于今天抱着巨大的勇气翻开了包志强老师的《点集拓扑和代数拓扑引论》,被文中老师幽默的行文,深深折服了,似乎拓扑也并没有想象中那么难,我想这是还不错的开始,我的第一直观感受拓扑也是给定一堆对象,在上面用一些公理弄些不一样的结构和代数一样,但是我暂时还不知道这堆结构要拿来干什么?有什么有趣的性质?

好吧,前面的路还很长,路漫漫,不过一想到前路那些绮丽的景色,多少还是有些兴奋的!虽然这路上没有一起分享喜悦的人...

2020年11月29日23:12:17

\section{Topology Space}
\subsection{Definition of Topology Space} 

\begin{itemize}
	\item 对于开集$U$的理解,首先它是$X$的子集,并且对于$\forall x \in U$,存在$x$的邻域包含于$U$。包老师的书里解释为$U$是每一个$x$的邻域,我感觉在这里似乎有点强了。wiki上解释为实数轴上的开区间的一般性推广。
	\item 那邻域是什么?在邻域之前,应该先理解基准开邻域结构(base open neighborhood),像定义代数结构一样,基准开邻域结构是一个映射$\func{\mathcal{N}}{X}{2^(2^X)}$,把每一个点$x \in X$对应到一个子集族$\mathcal{N}(x)$上,满足下面几条公理
		\begin{itemize}
			\item $\forall x \in X,\mathcal{N}(x) \neq \emptyset,$并且$\forall U \in \mathcal{U}(x),x \in U$
			\item 若$U,V \in \mathcal{U}(x),$则存在$W \in \mathcal{U}(x)$,使得$W = U \cap V$
			\item 若$y \in U \in \mathcal{U}(x)$,则存在$V \in \mathcal{N}(y)$,使得$V \subseteq U$
		\end{itemize}
\end{itemize}


\end{document}